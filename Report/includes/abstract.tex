 \begin{abstract}

\indent Image processing is a field that has many applications in life. It could be useful to extract the information and knowledge about phenomena as medical or biological processes. To obtain the best result, most of the applications must follow two processes: firstly, pre-processing the images with some appropriate operators to enhance the interest and to reduce the noises. Secondly, applying the measures or classification procedures to obtain the main results.\\[0.2cm]
\indent The goal of my internship at LaBRI is to build a fully functional program including all algorithms prensented in the article: \textbf{``Automatic identification of landmarks in digital images"}, which was proposed by Palaniswamy \textit{(IET Computer Vision 4.4(2010): 247-260)}. Besides that, several pre-processing operations have been added to facilitate the applications of these algorithms to a set of 2 dimensions images of beetles (293 animals have been analysed) provided by a team of biologists (INRA, Rennes).\\[0.2cm]
\indent At the end of the internship, I integrated the implementation of the article into the Image Processing for Morphometrics (IPM) software, which was developed previously by NGUYEN Hoang Thao (another PUF Master Degree). Besides, we also debug the previous code and write the documentation to help maintenance and addition of new operations.
\end{abstract}